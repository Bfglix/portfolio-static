\documentclass[11pt]{article}
\usepackage{graphicx} % Required for inserting images
\setlength{\parindent}{0pt}
\usepackage{hyperref}
\usepackage{enumitem}
\usepackage[utf8]{inputenc}
\usepackage[T1]{fontenc}
\usepackage[brazil]{babel}
\usepackage{lipsum}
\usepackage[left=1.06cm,top=1.7cm,right=1.06cm,bottom=0.49cm]{geometry}
\usepackage{makecell} % Added for table formatting in Skills section

\begin{document}
\begin{center}
    BRAULIO FELIX GOMEZ, Ph.D.\\
    \hrulefill
\end{center}

\begin{center}
    brauliofenixgomez@hotmail.com \textbullet \ LinkedIn: \href{https://www.linkedin.com/in/BraulioFelix}{Braulio Felix}
\end{center}

\vspace{12pt}

\begin{center}
    \textbf{About Me}
\end{center}
\includegraphics[width=0.3\textwidth]{placeholder_profile.jpg}
Hello! I am a passionate Robotics and AI Engineer with a Ph.D. from SUTD. I love building autonomous systems that solve real-world problems. My expertise spans across navigation stack development, computer vision, and machine learning. In my free time, I enjoy participating in robotics competitions and teaching.
\vspace{12pt}

\begin{center}
    \textbf{Education}
\end{center}
\textbf{Singapore University of Technology and Design (SUTD)} \hfill Singapore

Ph.D., Robotics, Automation, AI.
\hfill 2021 – 2025
\begin{itemize}[noitemsep, topsep=0pt, partopsep=0pt, parsep=0pt]
    \item \textit{Thesis Title:} Towards Effective Autonomy Strategies for Outdoor Robots.
\end{itemize}
I joined the PhD program in the Faculty of Engineering Product Development, where for four years conducted research on navigation algorithms for outdoor robotic systems. These algorithms were implemented and deployed across multiple robotic platforms. The resulting work was documented and disseminated through several peer-reviewed publications authored by me in collaboration with other researchers from SUTD. Each project I was involved in was funded by Singaporean government organizations as the National Robotics Programme.

\vspace{12pt}

\textbf{Instituto Tecnológico de Los Mochis} \hfill Mexico

BSc. Engineering in Informatics.
\hfill 2014 – 2019
\begin{itemize}[noitemsep, topsep=0pt, partopsep=0pt, parsep=0pt]
    \item \textit{Thesis Title:} Management System for University Student Services.
\end{itemize}
I enrolled in the Bachelor’s program in Informatics, where I developed intermediate to advanced programming skills in Java, JavaScript, C, and C++. Informatics is a discipline focused on how data is managed within backend systems and transformed into actionable information for decision-making in business environments and other automated software systems. For my graduation thesis, I programmed a content management system to support the administration of scholarships for international students.
\vspace{12pt}

\textbf{Centro de Bachillerato Tecnológico Industrial y de Servicios no.43} \hfill Mexico

Dip.Tech., Mechatronics.
\hfill 2010 – 2013

\vspace{12pt}

\begin{center}
    \textbf{Experience}
\end{center}
\textbf{Singapore University of Technology and Design, ROAR lab} \hfill Singapore

\textbf{PhD Candidate} \hfill 2021 – 2025
\begin{itemize}[noitemsep, topsep=0pt, partopsep=0pt, parsep=0pt]
    \item Led a robot autonomy team for the Panthera v2.0 outdoor sweeper robot.
    \item Developed a full navigation stack using the Robot Operating System (ROS) for the Panthera platform. Using for this C++ and Python3 and different control architectures like Behaviour Trees, Machine States, Reinforcement Learning , Genetic Algorithms and other Bio-Inspired decision making algorithms.
    \item Achieved LTA certification in Singapore for the Panthera platform after passing the T1 CETRAN test in two different dates for two different platforms, PantheraV1 and PantheraV2.
    \item Created a virtual robotics contest platform named Smorphi Imaginary, using the Unity engine, and a back-end programmed in NodeJS.Using for this C\#, Javascript, Typescript, HTML5,MongoDB.
    \item Deployed the Smorphi Imaginary platform for an official contest in March 2023 at Singapore, supporting more than 200 participants simultaneously.
    \item Developed an embodied AI framework for the Dragonfly Mosquito Catcher robot, reducing the complexity curve of deployment tasks. Using Python, chat GPT 5 official API and ROS
    \item Engineered a Next.js and WebSockets platform for the Dragonfly robot, using Typescript, Javascript , HTML5, CSS and MongoDB for data storage that enables the robotic platform to perform scheduled path planning, video streaming, remote operation and reporting of every sensor used by the robot.
    \item Led the publication of multiple papers in $Q_{1}$ journals related to AI, Machine Learning, Robotics, and Autonomy.
\end{itemize}

\vspace{12pt}

\textbf{LionsBot} \hfill Singapore

\textbf{AI/Robotics Enginner} \hfill 2020 – 2021
\begin{itemize}[noitemsep, topsep=0pt, partopsep=0pt, parsep=0pt]
    \item Supported research projects focused on robotics and automation, contributing to a body of work that resulted in multiple journal publications.
    \item Contributed at Lionsbot developing different computer vision modules and Reinforcement Learning solutions  for their robotic platforms, using for this Tensorflow, Pytorch and Intel OpenVINO.
    \item Contributed at Lionsbot developing different computer vision modules and Reinforcement Learning solutions  for their robotic platforms, using for this Tensorflow, Pytorch and Intel OpenVINO.
    \item Designed and implemented firmware in C/C++ to interface hardware components with custom controller software.
    \item Designed and implemented simulation environments for robotic platforms using Gazebo running on top of ROS1/ROS2.
\end{itemize}

\vspace{12pt}

\textbf{Free Lancer} \hfill Mexico-Singapore

\textbf{Full Stack Software Engineer} \hfill 2019 – 2025
\begin{itemize}[noitemsep, topsep=0pt, partopsep=0pt, parsep=0pt]
    \item Developed the backend for a blockchain-based candy machine system for NFT sales. Using the Solana blockchain. 
    \item Integrated the system to connect to users' wallets and trigger an RNG-based generator to supply image NFTs on the Solana blockchain. Using for this NodeJs, with Next and MongoDB for data storage.
    \item Developed the backend API in Node.js for SomosOne a contest web application for breakdance performers in Mexico. contributed to the full backend API and the roster creation system for the SomosOne social platform.
    \item Served as an instructor at the Instituto Tecnológico de Los Mochis, delivering a training course for faculty titled “Modern Prototyping Techniques: 3D Model Design Focused on 3D Printing.”. Teaching in this course how to use SolidWorks to prototype mechanical parts and the design considerations to facilitate the 3D printing process.
    \item Developed data pipelines, inference pipelines, and training pipelines using libraries such as TensorFlow and PyTorch, deployed on cloud computing platforms including AWS, Google Cloud, and Azure. Demonstrated a strong understanding of AI governance and Model Risk Management.
\end{itemize}
\vspace{12pt}
\begin{center}
    \textbf{Certifications}
\end{center}
\begin{itemize}[noitemsep, topsep=0pt, partopsep=0pt, parsep=0pt]
    \item AI Engineer MLOps Track: Deploy Gen AI \& Agentic AI at Scale from Udemy (2025)
    \item Devops y cloud con azure devops, app service pipelines y git from Udemy (2025)
    \item Programming Patterns in JavaScript and TypeScript from Udemy (2025)
    \item Javascipt Intermediate from HackerRank (2025)
    \item Problem Solving Intermediate from HackerRank (2025)
    \item Rest API Intermediate from HackerRank (2025)
    \item Software Engineer from HackerRank (2025)
    \item SQL intermediate from HackerRank (2025)
\end{itemize}
\vspace{12pt}

\begin{center}
    \textbf{Skills}
\end{center}

\textbf{Coding (+4 years):} Java, PHP, Python, SQL, C/C++, XML/XSL, $\mathrm{LAT}_{\mathrm{E}} \mathrm{X}$

\textbf{AI (+5 years):} Experience developing data pipelines, inference pipelines and training pipelines using libraries like Tensorflow, Pytorch and cloud systems like AWS, Gcloud, Azure. Strong understanding in AI governance and Model Risk Management.

\textbf{Web Dev (+2 years):} HTML, CSS, JavaScript, Apache Web Server, NodeJS. Experience troubleshooting networking and system-level issues in distributed deployments.

\textbf{Databases (+5 years):} MYSQL, SQLITE, MongoDB

\textbf{Dev Ops (+1 years):} GITHUB, GITLAB, experience creating development workflows and coordinating developers, experience deploying Docker in production-like settings.

\textbf{Hardware: } Experience interfacing hardware to custom made controller software, experience in firmware programming using C/C++.

\textbf{Languages:} Strong reading, writing, and speaking competencies for English and Spanish

\textbf{Operating System Management:} Strong knowledge in Linux for deployment.

\textbf{Misc. Competencies:} Academic research (+5 years), teaching, training, and publishing

\vspace{12pt}



\begin{center}
    \textbf{Research Publications}
\end{center}

\textbf{Journal Articles}
\begin{itemize}[noitemsep, topsep=0pt, partopsep=0pt, parsep=0pt]
    \item B. F. Gómez, J. W. S. Lee, A. Jayadeep, M. A. V. J. Muthugala, and M. R. Elara, "Efficient Robot-Aided Outdoor Cleaning with a Glasius Bio-Inspired Neural Network and Vision-Based Adaptation," \textit{Mathematics}, vol. 13, no. 20, p. 3277, 2025. \url{https://www.mdpi.com/2227-7390/13/20/3277}
    \item Gómez, Braulio Félix, L. Yi, B. Ramalingam, et al., "Deep learning based litter identification and adaptive cleaning using self-reconfigurable pavement sweeping robot," in \textit{2022 IEEE 18th International Conference on Automation Science and Engineering (CASE)}, IEEE, 2022, pp. 2301-2306. \url{https://ieeexplore.ieee.org/document/9926489/}
    \item Gómez, Braulio Félix, A. Jayadeep, M. V. J. Muthugala, and M. R. Elara, "A framework for coverage path planning of outdoor sweeping robots deployed in large environments," \textit{Mathematics}, vol. 13, no. 14, p. 2238, 2025. \url{https://www.mdpi.com/2227-7390/13/14/2238}
    \item S. Pookkuttath, Gomez, Braulio Felix, and M. R. Elara, "RI-based vibration-aware path planning for mobile robots' health and safety," \textit{Mathematics}, vol. 13, no. 6, p. 913, 2025. \url{https://www.mdpi.com/2227-7390/13/6/913}
    \item J. H. Ong, A. A. Hayat, Gomez, Braulio Felix, M. R. Elara, and K. L. Wood, "Deep learning based fall recognition and forecasting for reconfigurable stair-accessing service robots," \textit{Mathematics}, vol. 12, no. 9, p. 1312, 2024. \url{https://www.mdpi.com/2227-7390/12/9/1312}
    \item S. Pookkuttath, Gomez, Braulio Felix, M. R. Elara, and P. Thejus, "An optical flow-based method for condition-based maintenance and operational safety in autonomous cleaning robots," \textit{Expert Systems with Applications}, vol. 222, p. 119 802, 2023. \url{https://doi.org/10.1016/j.eswa.2023.119802}
    \item T. Pathmakumar, M. V. J. Muthugala, S. B. P. Samarakoon, Gómez, Braulio Félix, and M. R. Elara, "A novel path planning strategy for a cleaning audit robot using geometrical features and swarm algorithms," \textit{Sensors}, vol. 22, no. 14, p. 5317, 2022. \url{https://www.mdpi.com/1424-8220/22/14/5317}
    \item L. Yi, Félix Gómez, Braulio, B. Ramalingam, M. M. Rayguru, M. R. Elara, and A. A. Hayat, "Self-reconfigurable robot vision pipeline for safer adaptation to varying pavements width and surface conditions," \textit{Scientific reports}, vol. 12, no. 1, p. 14 557, 2022. \url{https://www.nature.com/articles/s41598-022-17858-w}
    \item A. V. Le, B. Ramalingam, Gomez, Braulio Felix, R. E. Mohan, T. H. Q. Minh, and V. Sivanantham, "Social density monitoring toward selective cleaning by human support robot with 3d based perception system," \textit{IEEE Access}, vol. 9, pp. 41 407-41 416, 2021. \url{https://ieeexplore.ieee.org/document/9366838}
    \item L. M. J. Melvin, R. E. Mohan, A. Semwal, et al., "Remote drain inspection framework using the convolutional neural network and re-configurable robot raptor," \textit{Scientific reports}, vol. 11, no. 1, p. 22 378, 2021. \url{https://www.nature.com/articles/s41598-021-01170-0}
    \item P. Palanisamy, R. E. Mohan, A. Semwal, et al., "Drain structural defect detection and mapping using ai-enabled reconfigurable robot raptor and iort framework," \textit{Sensors}, vol. 21, no. 21, p. 7287, 2021. \url{https://www.mdpi.com/1424-8220/21/21/7287}
    \item T. Pathmakumar, M. R. Elara, Gómez, Braulio Félix, and B. Ramalingam, "A reinforcement learning based dirt-exploration for cleaning-auditing robot," \textit{Sensors}, vol. 21, no. 24, p. 8331, 2021. \url{https://www.mdpi.com/1424-8220/21/24/8331}
    \item B. Ramalingam, R. Elara Mohan, S. Balakrishnan, et al., "Stetro-deep learning powered staircase cleaning and maintenance reconfigurable robot," \textit{Sensors}, vol. 21, no. 18, p. 6279, 2021. \url{https://www.mdpi.com/1424-8220/21/18/6279}
    \item B. Ramalingam, A. A. Hayat, M. R. Elara, et al., "Deep learning based pavement inspection using self-reconfigurable robot," \textit{Sensors}, vol. 21, no. 8, p. 2595, 2021. \url{https://www.mdpi.com/1424-8220/21/8/2595}
    \item B. Ramalingam, T. Tun, R. E. Mohan, et al., "Ai enabled iort framework for rodent activity monitoring in a false ceiling environment," \textit{Sensors}, vol. 21, no. 16, p. 5326, 2021. \url{https://www.mdpi.com/1424-8220/21/16/5326}
    \item L. Yi, A. V. Le, B. Ramalingam, et al., "Locomotion with pedestrian aware from perception sensor by pavement sweeping reconfigurable robot," \textit{Sensors}, vol. 21, no. 5, p. 1745, 2021. \url{https://www.mdpi.com/1424-8220/21/5/1745}
    \item B. Ramalingam, R. E. Mohan, S. Pookkuttath, et al., "Remote insects trap monitoring system using deep learning framework and iot," \textit{Sensors}, vol. 20, no. 18, p. 5280, 2020. \url{https://www.mdpi.com/1424-8220/20/18/5280}
    \item B. Ramalingam, J. Yin, M. Rajesh Elara, et al., "A human support robot for the cleaning and maintenance of door handles using a deep-learning framework," \textit{Sensors}, vol. 20, no. 12, p. 3543, 2020. \url{https://www.mdpi.com/1424-8220/20/12/3543}
    \item M. M. Rayguru, M. R. Elara, Gomez, Braulio Felix, and B. Ramalingam, "A time delay estimation based adaptive sliding mode strategy for hybrid impedance control," \textit{IEEE Access}, vol. 8, pp. 155 352-155 361, 2020. \url{https://ieeexplore.ieee.org/document/9153920}
    \item T. W. Teng, P. Veerajagadheswar, B. Ramalingam, J. Yin, R. Elara Mohan, and Gómez, Braulio Félix, "Vision based wall following framework: A case study with hsr robot for cleaning application," \textit{Sensors}, vol. 20, no. 11, p. 3298, 2020. \url{https://www.mdpi.com/1424-8220/20/11/3298}
\end{itemize}

\textbf{Conference Proceedings}
\begin{itemize}[noitemsep, topsep=0pt, partopsep=0pt, parsep=0pt]
    \item Gómez, Braulio Félix, L. Yi, B. Ramalingam, et al., "Deep learning based litter identification and adaptive cleaning using self-reconfigurable pavement sweeping robot," in \textit{2022 IEEE 18th International Conference on Automation Science and Engineering (CASE)}, IEEE, 2022, pp. 2301-2306.
\end{itemize}